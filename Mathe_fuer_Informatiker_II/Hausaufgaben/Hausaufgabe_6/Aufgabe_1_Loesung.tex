Bestimmen der Eigenwerte und Eigenräume der Matrix
\[
A = \begin{pmatrix}
-5 & 0 & 7 \\
6 & 2 & -6 \\
-4 & 0 & 6
\end{pmatrix}
\]

1. Charakteristisches Polynom

Das charakteristische Polynom \( p(\lambda) \) einer Matrix \( A \) wird durch die Determinante der Matrix \( A - \lambda I \) berechnet:
\[
\det(A - \lambda I) = 0
\]

Hier ist \( I \) die Einheitsmatrix:
\[
A - \lambda I = \begin{pmatrix}
-5 - \lambda & 0 & 7 \\
6 & 2 - \lambda & -6 \\
-4 & 0 & 6 - \lambda
\end{pmatrix}
\]

Die Determinante der Matrix \( A - \lambda I \) ist:
\[
\det(A - \lambda I) = \begin{vmatrix}
-5 - \lambda & 0 & 7 \\
6 & 2 - \lambda & -6 \\
-4 & 0 & 6 - \lambda
\end{vmatrix}
\]

Die Berechnung der Determinante ergibt:
\[
\det(A - \lambda I) = (-5 - \lambda) \begin{vmatrix}
2 - \lambda & -6 \\
0 & 6 - \lambda
\end{vmatrix} + 7 \begin{vmatrix}
6 & 2 - \lambda \\
-4 & 0
\end{vmatrix}
\]

\[
= (-5 - \lambda) [(2 - \lambda)(6 - \lambda) - 0] + 7 [6 \cdot 0 - (-4)(2 - \lambda)]
\]

\[
= (-5 - \lambda) [(2 - \lambda)(6 - \lambda)] + 7 [4(2 - \lambda)]
\]

\[
= (-5 - \lambda) [12 - 8\lambda + \lambda^2] + 7 [8 - 4\lambda]
\]

\[
= -60 + 40\lambda - 5\lambda^2 - 12\lambda + 8\lambda^2 + 56 - 28\lambda
\]

\[
= 3\lambda^2 - \lambda - 4
\]

Das charakteristische Polynom ist also:
\[
p(\lambda) = 3\lambda^2 - \lambda - 4
\]

2. Eigenwerte

Um die Eigenwerte zu finden, wird das charakteristische Polynom gleich null gesetzt:
\[
3\lambda^2 - \lambda - 4 = 0
\]

Lösen dieser quadratischen Gleichung ergibt die Eigenwerte:
\[
\lambda_1 = 2, \quad \lambda_2 = -\frac{2}{3}
\]

3. Eigenvektoren

Für \(\lambda_1 = 2\):
\[
A - 2I = \begin{pmatrix}
-7 & 0 & 7 \\
6 & 0 & -6 \\
-4 & 0 & 4
\end{pmatrix}
\]

Lösen des Gleichungssystems \((A - 2I)\mathbf{v} = 0\) ergibt den Eigenvektor:
\[
\mathbf{v_1} = \begin{pmatrix}
1 \\
0 \\
1
\end{pmatrix}
\]

Für \(\lambda_2 = -\frac{2}{3}\):
\[
A + \frac{2}{3}I = \begin{pmatrix}
-5 + \frac{2}{3} & 0 & 7 \\
6 & 2 + \frac{2}{3} & -6 \\
-4 & 0 & 6 + \frac{2}{3}
\end{pmatrix}
= \begin{pmatrix}
-\frac{13}{3} & 0 & 7 \\
6 & \frac{8}{3} & -6 \\
-4 & 0 & \frac{20}{3}
\end{pmatrix}
\]

Lösen des Gleichungssystems \((A + \frac{2}{3}I)\mathbf{v} = 0\) ergibt den Eigenvektor:
\[
\mathbf{v_2} = \begin{pmatrix}
0 \\
1 \\
0
\end{pmatrix}
\]
