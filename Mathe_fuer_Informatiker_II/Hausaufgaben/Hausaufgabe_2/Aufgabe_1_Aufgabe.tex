Im Raum $\mathbb{R}^4$ sei die Ebene $\Gamma = \vec{r_0} + \lambda_1 \vec{a_1} + \lambda_2 \vec{a_2}$, sowie der Punkt $\vec{r_1}$ gegeben mit

\[
\vec{r_0} = \begin{pmatrix} 0 \\ 1 \\ 2 \\ 2 \end{pmatrix}, \quad
\vec{r_1} = \begin{pmatrix} 1 \\ 3 \\ 5 \end{pmatrix}, \quad
\vec{a_1} = \begin{pmatrix} 1 \\ 1 \\ 0 \end{pmatrix}, \quad
\vec{a_2} = \begin{pmatrix} 0 \\ -1 \\ 1 \end{pmatrix}
\]

Man gebe den Abstand $d(\vec{r_1}, \Gamma)$ von $\vec{r_1}$ und $\Gamma$ an und bestimme den Fußpunkt $\vec{r_F}$ des Lotes von $\vec{r_1}$ auf $\Gamma$.