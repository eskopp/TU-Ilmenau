Gegeben sei die Menge \( M = \{0, 1, 2\}^n \) der \( n \)-dimensionalen Zeilenvektoren (Worte) mit Einträgen 0, 1 oder 2. Auf \( M \) wird ein Abstandsmaß \( d \colon M^2 \to \mathbb{R} \) definiert durch
\[ d(a, b) = \text{Anzahl der Stellen, an denen sich } a \text{ und } b \text{ unterscheiden} \]

Bsp: \( d((1, 1, 1, 0), (2, 1, 1, 3)) = 2 \). Zeigen Sie, dass \( d \) eine Metrik ist, d.h. dass die Eigenschaften (D1) bis (D3) erfüllt sind.

(Bemerkung: Man nennt \( d(a, b) \) den Hammingabstand der Worte \( a \) und \( b \).)
