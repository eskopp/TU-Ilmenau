Gegeben seien die folgenden beiden Normen im $\mathbb{R}^2$.

\[
\left\| \begin{pmatrix} x_1 \\ x_2 \end{pmatrix} \right\|_1 = |x_1| + |x_2| \quad \text{(Betragssummennorm/Manhattan-Norm/1-Norm)}
\]

\[
\left\| \begin{pmatrix} x_1 \\ x_2 \end{pmatrix} \right\|_\infty = \max\{|x_1|, |x_2|\} \quad \text{(Maximumnorm/Unendlich-Norm)}
\]

Die zugehörigen Metriken erhält man durch $d(\mathbf{x}, \mathbf{y}) = \|\mathbf{x} - \mathbf{y}\|$.

Der Einheitskreis ist die Menge $K = \{\mathbf{x} \mid d(\mathbf{x}, \mathbf{0}) = 1\}$ aller Punkte, die den Abstand 1 vom Koordinatenursprung haben. Zeichnen Sie in der Ebene die Einheitskreise bezüglich der euklidischen Metrik, der Manhattanmetrik und der Maximummetrik ein.
