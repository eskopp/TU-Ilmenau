
\section*{Berechnung des Ausgleichspolynoms zweiten Grades}

Gegeben sind die Punktpaare:
\[
(x_1, y_1) = (-2, 3), \quad (x_2, y_2) = (-1, 1), \quad (x_3, y_3) = (0, 0), \quad (x_4, y_4) = (1, 3), \quad (x_5, y_5) = (2, 6)
\]

Das gesuchte Ausgleichspolynom ist definiert als:
\[
p(x) = a_0 + a_1 x + a_2 x^2
\]

\subsection*{Aufstellung der Design-Matrix und des Vektors}
Die Design-Matrix \( A \) und der Vektor \( \mathbf{y} \) sind:
\[
A = \begin{pmatrix}
1 & -2 & 4 \\
1 & -1 & 1 \\
1 & 0 & 0 \\
1 & 1 & 1 \\
1 & 2 & 4
\end{pmatrix}, \quad \mathbf{y} = \begin{pmatrix}
3 \\
1 \\
0 \\
3 \\
6
\end{pmatrix}
\]

\subsection*{Formulierung der Normalgleichungen}
Die Normalgleichungen \( A^T A \mathbf{a} = A^T \mathbf{y} \) sind:
\[
A^T A = \begin{pmatrix}
5 & 0 & 10 \\
0 & 10 & 0 \\
10 & 0 & 30
\end{pmatrix}, \quad A^T \mathbf{y} = \begin{pmatrix}
13 \\
8 \\
28
\end{pmatrix}
\]

\subsection*{Berechnung der inversen Matrix von \(A^T A\)}
Die Inverse von \(A^T A\) berechnen wir durch:
\[
(A^T A)^{-1} = \frac{1}{\text{det}(A^T A)} \cdot \text{adj}(A^T A)
\]
wobei \(\text{det}(A^T A)\) die Determinante von \(A^T A\) und \(\text{adj}(A^T A)\) die Adjunkte von \(A^T A\) ist.

\subsection*{Lösung des linearen Gleichungssystems}
Das lineare Gleichungssystem \( A^T A \mathbf{a} = A^T \mathbf{y} \) lösen wir durch:
\[
\mathbf{a} = (A^T A)^{-1} A^T \mathbf{y}
\]
wobei die Koeffizienten \(\mathbf{a}\) gefunden werden durch Multiplikation der Inversen von \(A^T A\) mit \(A^T \mathbf{y}\), was ergibt:
\[
\mathbf{a} = \begin{pmatrix}
0.6 \\
0.8 \\
1.0
\end{pmatrix}
\]

Das Ausgleichspolynom lautet somit:
\[
p(x) = 0.6 + 0.8x + x^2
\]
